\documentclass[UTF8]{ctexart}
\usepackage{amsmath}
\usepackage{diagbox}
\usepackage{textcomp}
\usepackage{graphicx}
\usepackage{float}
\usepackage{caption}
\usepackage{adjustbox}
\usepackage{subfigure}
\usepackage{geometry}
\usepackage{pifont}
\usepackage{gensymb}
\usepackage{bm}
\usepackage{tikz}
\usepackage{amstext}
\usepackage{amsfonts}

%引入代码块
\usepackage{listings}

\usepackage{xcolor}
%设置代码块格式

\definecolor{CPPGray}{RGB}{211,211,211}
\lstset{
 columns=fixed,       
 numbers=left,   % 在左侧显示行号
 numberstyle=\tiny\color{gray},% 设定行号格式
 frame=none,%none,% 不显示背景边框
 %aboveskip=1em,
 backgroundcolor=\color[RGB]{230,230,230},% 设定背景颜色
 keywordstyle=\color[RGB]{40,40,255},% 设定关键字颜色
 numberstyle=\footnotesize\color{darkgray},           
 commentstyle=\it\color[RGB]{0,96,96},% 设置代码注释的格式
 stringstyle=\rmfamily\slshape\color[RGB]{128,0,0},% 设置字符串格式
 showstringspaces=true,% 不显示字符串中的空格
 language=c++, % 设置语言
 morekeywords = {include,ull,int,double,return,static,typedef,if,else,for,long,void,class,struct,ll},                % 自加新的关键字(必须前后都是空格)
}

\begin{document}
\renewcommand{\thefootnote}{\fnsymbol{footnote}}
\newgeometry{left=2cm,bottom=3cm,right=2cm}
\linespread{1.4}
\title{\vspace{-5em}\heiti算法分析与设计基础\ \ 第十周作业\vspace{-2.5em}}
\date{}
\maketitle
\begin{center}
{\fangsong 徐浩博\quad 软件02\quad2020010108}
\end{center}

\paragraph*{Problem\ \ 1}\ \par
我们采用势函数$\Phi(T)=|2\cdot T.num - T.size|$,来看TABEL-DELETE的操作摊还代价.\par
i)当$\alpha_{i-1}<1/2$,且DELETE操作不会引起表的收缩,则$c_i=1$,$\alpha_{i}<1/2$,有:
\begin{align*}
    \hat{c_i}&=c_i+\Phi_i-\Phi_{i-1}\\
    &=1+|2\cdot num_i - size_i|-|2\cdot num_{i-1} - size_{i-1}|\\
    &=1+size_i-2\cdot num_i-size_{i-1}+2\cdot num_{i-1}\\
    &=1+(size_i-size_{i-1})-2( num_i-num_{i-1})\\
    &=1+1-0=2
\end{align*}\par
ii)当$\alpha_{i-1}<1/2$,且DELETE操作会引起表的收缩,则$c_i=num_{i}+1$(新建slot并且复制原有元素),$size_{i} = 2size_{i-1}/3$(size收缩),$num_{i}< size_{i-1}/3$(触发收缩的条件)有:
\begin{align*}
    \hat{c_i}&=c_i+\Phi_i-\Phi_{i-1}\\
    &=num_{i}+1+|2\cdot num_i - size_i|-|2\cdot num_{i-1} - size_{i-1}|\\
    &=num_i+1+size_i-2num_i-size_{i-1}+2num_{i-1}\\
    &=1+(size_i-size_{i-1})-num_i+2num_{i-1}\\
    &=1-size_{i-1}/3-num_i+2num_{i-1}\\
    &<1-num_i-num_i+2num_{i-1}\\
    &=1+2(num_{i-1}-num_i)\\
    &=1+2=3
\end{align*}\par 
iii)当$\alpha_{i-1}\geq 1/2$,即$num_{i-1}\geq size_{i-1}/2$.\par 若DELETE引起表收缩,则说明$num_{i}=num_{i-1}-1< size_{i-1}/3$,结合两个不等式,有$ size_{i-1}/2 - 1<size_{i-1}/3$,推出$size_{i-1}<6$,由此我们进行如下较为宽松的放缩证明摊还代价上界是个常数:
\begin{align*}
    \hat{c_i}&=c_i+\Phi_i-\Phi_{i-1}\\
    &=num_{i}+1+|2\cdot num_i - size_i|-|2\cdot num_{i-1} - size_{i-1}|\\
    &<size_{i-1}+1+(2\cdot num_i + size_i)+0\\
    &<size_{i-1}+1+2size_{i-1}+size_{i-1}\\
    &<6+1+12+6=25
\end{align*}\par 
若DELETE不引起表收缩,且$2\cdot num_i - size_i<0$,那么考虑到$num_{i-1}\geq size_{i-1}/2$,则有$|2\cdot num_i - size_i|=|2\cdot (num_{i-1}-1) - size_{i-1}|\geq 2$,,故该绝对值式子的最大值为2,有
\begin{align*}
    \hat{c_i}&=c_i+\Phi_i-\Phi_{i-1}\\
    &=1+|2\cdot num_i - size_i|-|2\cdot num_{i-1} - size_{i-1}|\\
    &=1+2-2num_{i-1}+size_{i-1}\\
    &=3-2(num_{i-1}+size_{i-1}/2)\\
    &\leq 3
\end{align*}\par 
若DELETE不引起表收缩,且$2\cdot num_i - size_i\geq 0$,那么:
\begin{align*}
    \hat{c_i}&=c_i+\Phi_i-\Phi_{i-1}\\
    &=1+|2\cdot num_i - size_i|-|2\cdot num_{i-1} - size_{i-1}|\\
    &=1+2num_i-size_i-2num_{i-1}+size_{i-1}\\
    &=1-2(num_{i-1}-num_i)+(size_{i-1}-size_i)\\
    &=1-2+0=-1
\end{align*}\par
综合以上,TABLE\_DELETE操作的摊还代价上界是个常数.
\paragraph*{Problem\ \ 2}\ \par
\textbf{a. SEARCH}\par
\begin{lstlisting}[mathescape=true]
SEARCH(A, n, key):
    for i from k - 1 to 0:
        if $A_i$ is not empty (i.e. $n_i == 1$):
            result = BINARY_SEARCH(A_i, key)
            if result != -1:
                return i, result
    return -1, -1

BINARY_SEARCH(a, key):
    low = 0, high = a.size - 1
    while low <= high:
        mid = (low + high) / 2
        if a[mid] == key:
            return key
        if a[mid] < key:
            low = mid + 1
        else:
            high = mid - 1
    return -1
\end{lstlisting}\par
首先,BINARY\_SEARCH,也即二分查找的最坏复杂度为O(log(length)),其中length是该数组的长度. 这些有序数组均没有找到该数是一种最差情况,且其他的情况不会差于此情况;在这种情况下将$A_0\cdots A_{k-1}$的非空数组均进行了二分查找,那么显然,这些数组均非空时时间开销最大,在这种情况下时间开销$T(n)=\sum_{i=0}^k log(A_i.size)=1+2+\cdots+(log(n)-1)=(log(n)-1)log(n)/2=O(log^2n)$.\par
\textbf{b. INSERT}\par
\begin{lstlisting}[mathescape=true]
INSERT(A, k, key):
    i = 0
    let aux be an auxiliary array    
    while $A_i$ is full (i.e. $n_i==1$):
        i = i + 1
    if i == k:
        k = k + 1
    aux[0] = key
    for j from 0 to i - 1:
        merge($A_j$, aux, aux)
        clear all the elements in $A_j$
    copy aux to $A_i$
    return

merge(a, b, result):
    length1 = length of a
    length2 = length of b
    i = 0, j = 0, k = 0
    let aux be an auxiliary array    
    while i < length1 and j < length2:
        if a[i] < b[j]:
            aux[k] = a[i]
            k = k + 1
            i = i + 1
        else:
            aux[k] = b[j]
            k = k + 1
            j = j + 1
    while i < length1:
        aux[k] = a[i]
        k = k + 1
        i = i + 1
    while j < length2:
        aux[k] = b[j]
        k = k + 1
        j = j + 1
    copy aux to result
    return result
\end{lstlisting}
我们先分析进位时若干$A_0\cdots A_{m-1}$合并的时间开销. 可以看到,我们的策略是从$A_0$开始向上合并;每调用一次merge合并的时间开销线性于a,b两个数组的长度,记为O(c·length),那么合并这i个数组的时间开销是$\sum_{i=1}^{m-1}c(\sum_{j=0}^{i-1}length(A_j)+length(A_i))=\sum_{i=1}^{m-1}c(2^{i-1}+c^i)<c(m2^0+(m-1)2^1+\cdots+1\cdot 2^{m-1})<c2^{m+1}=O(2^m)$.\par
最坏情况:显然是所有的低位均需进位,$m=\lfloor{log_2(n+1)}\rfloor$,则合并数组的开销就是$O(2^m)=O(n)$.\par
摊还分析:我们采用聚集法进行分析,对于每一次进位合并,时间开销为O($2^m$)正比于m个数组的总元素个数,因此我们可以认为每次进位,均摊到每个元素上的时间开销都是O(1)的,对于$A_i$来说,复杂度就是$c2^i$. 下面我们对每个数组$A_i$分析,它每次需要合并时的时间开销均是$c2^i$,共需进位$\lfloor\frac{n}{2^i}\rfloor$次,因此插入n个元素时,把所有的$A_i$时间开销相加,得:$\sum_{i=0}^{k-1}c2^i\lfloor\frac{n}{2^i}\rfloor<c\sum_{i=0}^{k-1}n=cnk=cn\lfloor log(n+1)\rfloor=O(nlogn)$. 每个操作的平均代价就是O(nlogn/n)=O(logn).\par
\textbf{c. DELETE}\par
\begin{lstlisting}[mathescape = true]
DELETE(A, n, key):
    i, pos = SEARCH(A, n ,key)
    if(i == -1) return
    n = n - 1
    j = 0
    while $A_j$ is empty (i.e. $n_j == 0$):
        j++
    if i != j:
        SPLIT(A, j, 0)
        k = pos
        while k > 0:
            if $A_i$[k - 1] <= $A_j$[0]:
                $A_i$[k] = $A_j$[0]
                break
            else:
                $A_i$[k] = $A_i$[k - 1]
            k = k - 1
        if k == 0:
            $A_i$[k] = $A_j$[0]
    else:
    SPLIT(A, j ,pos)

SPLIT(A, i, pos):
    j = i - 1
    k = 0
    while $A_j$ is empty (i.e. $n_j == 0$):
        size = $2^j$
        if k + size - 1 < pos or k > pos:
            copy $A_i$[k $\cdots$ k + size - 1] to $A_j$
        else 
            copy $A_i$[k $\cdots$ pos - 1] and $A_i$[pos + 1 $\cdots$ k + pos] to $A_j$
\end{lstlisting}\par
算法可以总结为:先搜索要删除的元素位置,这一步是O(logn)的. 假设要删除的元素在$A_i$内,而$A_j$为j最小的非空数组,那么删除时需要将$A_j$一个元素给$A_i$(此处我们给出$A_j[0]$),将$A_j[0]$在$A_i$中安插好,然后将$A_j$分裂成$A_0\cdots A_{j-1}$. 一种特殊的情况是$i == j$,则直接将$A_i$分裂即可. 考虑到数组均是有序的,因此无论是安插$A_j[0]$还是分裂$A_j$,都和元素个数是呈线性的. 考虑到$A_j$元素不多于$A_i$,因此删除操作的复杂度近似和$A_i$的元素个数成正比,为$O(2^i)$.\par
下面我们估算删除已有的每个元素平均用时:对于$A_i$,他有$2^i$个元素,因此删除的元素在$2^i$内的概率为$2^i/2^k$,删除的时间开销为$c2^i$. 因此计算平均时间开销有$\sum_{i=0}^{k-1}c2^i\times p(A_i)=c/2^k\cdot\sum_{i=0}^{k-1}2^{2i}<c2^k=O(2^{logn})=O(n)$. 即平均时间开销是O(n)的.
\end{document}

