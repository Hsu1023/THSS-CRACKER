\documentclass[UTF8]{ctexart}
\usepackage{amsmath}
\usepackage{diagbox}
\usepackage{textcomp}
\usepackage{graphicx}
\usepackage{float}
\usepackage{caption}
\usepackage{adjustbox}
\usepackage{subfigure}
\usepackage{geometry}
\usepackage{pifont}
\usepackage{gensymb}
\usepackage{bm}
\usepackage{tikz}
\usepackage{amstext}
\usepackage{amsfonts}

%引入代码块
\usepackage{listings}

\usepackage{xcolor}
%设置代码块格式

\definecolor{CPPGray}{RGB}{211,211,211}
\lstset{
 columns=fixed,       
 numbers=left,   % 在左侧显示行号
 numberstyle=\tiny\color{gray},% 设定行号格式
 frame=none,%none,% 不显示背景边框
 %aboveskip=1em,
 backgroundcolor=\color[RGB]{230,230,230},% 设定背景颜色
 keywordstyle=\color[RGB]{40,40,255},% 设定关键字颜色
 numberstyle=\footnotesize\color{darkgray},           
 commentstyle=\it\color[RGB]{0,96,96},% 设置代码注释的格式
 stringstyle=\rmfamily\slshape\color[RGB]{128,0,0},% 设置字符串格式
 showstringspaces=true,% 不显示字符串中的空格
 language=c++, % 设置语言
 morekeywords = {include,ull,int,double,return,static,typedef,if,else,for,long,void,class,struct,ll},                % 自加新的关键字(必须前后都是空格)
}

\begin{document}
\renewcommand{\thefootnote}{\fnsymbol{footnote}}
\newgeometry{left=2cm,bottom=3cm,right=2cm}
\linespread{1.4}
\title{\vspace{-5em}\heiti算法分析与设计基础\ \ 第九周作业\vspace{-2.5em}}
\date{}
\maketitle
\begin{center}
{\fangsong 徐浩博\quad 软件02\quad2020010108}
\end{center}

\paragraph*{Problem\ \ 1}\ \par
首先我们分析Huffman编码的过程,对于一开始的256个字符,记它们的频率为$freq_1,\cdots,freq_{256}$. 由最高频率低于最低频率2倍得,$freq_i+freq_j\geq 2MIN\{freq\}>MAX\{freq\}\geq freq_k, \forall i,j,k$,因此初始这256个节点进行Huffman算法时,不存在任何已被合并的节点和未被合并的节点进行合并;对应到Huffman树上,这256个节点均是Huffman树T的叶子节点. 下面考虑它们合成的128个节点$freq_1',\cdots,freq_{128}'$. 我们依然有$freq_i'+freq_j'\geq 4MIN\{freq\}>2MAX\{freq\}\geq freq_k', \forall i',j',k'$,因此这128个节点也不会和二次合并的节点进行合并,对应到Huffman树上,表现为256个叶子节点的父节点两个两个构成兄弟. 依次类推,最终合并剩1个节点时,整个Huffman编码树是一个满二叉树,树高为$log_2256=8$,因此每个字符对应的前缀码的码长仍是8,这与固定编码的8位固定长度是一样的,所以对于这个数据文件来说,Huffman编码并不比8位固定长度编码更高效.

\paragraph*{Problem\ \ 2}\ \par
\textbf{a.} 
贪心算法如下:
\begin{lstlisting}
GREEDY_ALGORITHM(amount):
    let array A be the result
    count = 0 
    while(amount > 25):
        A[count] = 25
        count += 1
        amount -= 25
    while(amount > 10):
        A[count] = 10
        count += 1
        amount -= 10
    while(amount > 5):
        A[count] = 5
        count += 1
        amount -= 5
    while(amount > 1):
        A[count] = 1
        count += 1
        amount -= 1
    return A
\end{lstlisting}
采用\textbf{归纳法}证明贪心算法能够取到最优策略. \par
首先,n=1时,取贪心算法要求我们取1张1元硬币,则显然是最优策略.\par
假设$n<k$时贪心算法也正确,n=k时,我们分情况讨论:\par
i) $k\leq 4$,显然贪心地取k张1元硬币是最优策略.\par
ii) $5\leq k\leq 9$,只有两种策略:"k张1元"或者"1张5元+(k-5)张1元",后者明显更优,而贪心算法给出的也是后一种策略,因此贪心能取到最优策略.\par
iii) $10\leq k< 25$, 在最优策略中,1元的数量不应超过4,否则用1个五元替换5个一元能够获得更优解. 类似地,5元数量不应该超过1. 在这种情况下,1元和5元能够构成的最大面额为1元$\times$4+5元$\times$1=9元,要构成k元必须至少要一张10元,因此最优策略中肯定有10元. 我们先贪心地取10元,根据归纳假设,剩下$k-10$元也可以贪心地取得到最优策略,因此这k元都可以贪心地取到最优策略.
\par iv)$k\geq 25$. 在最优策略中,1元的数量不应超过4,否则用1个五元替换5个一元能够获得更优解. 类似地,5元数量不应该超过1,10元数量不应超过4(2个25元可以替代5个10元). 最优策略中没有25元,则1元和5元组合成的最大面额为9元(5元最多1张,1元最多4张),需要组成$k\geq 25$的情况仍需要至少2张10元;而至少2张10元意味着最优策略中没有5元(否则2张10元和1张5元可以用1张25元来替代);没有5元,而1元最多4张,则意味着10元至少需要3张才能使总额超过25元,而3张10元可以用1张25元和1张5元代替,则没有25元的策略不是最优策略,产生矛盾. 所以,$k>25$时最优策略必须有一张25元,先取这张25元,由归纳假设,剩下$k-25$元也可以贪心地取,因此,这k元用贪心算法可以得到最优策略. \par
由归纳法知,贪心算法求出的是最优策略.\par\ \ \par 

\textbf{b.} 对于每个需要找零的面值n,假设k是满足$c^k\leq n$的最大整数. \par 首先,在最优方案中,$c^0,\cdots,c^{k-1}$每个面额的硬币最多有$c-1$个. 这一点可以通过反证法证明:假设最优方案中$c^i$超过c-1个,那么c个$c^i$面额硬币可以用1个$c^{i+1}$面额硬币代替,而且这样替代能够使得方案中用的硬币数更少,与最优方案矛盾. 因此,最优方案中每个面额的硬币最多$c-1$个.\par
其次,我们要说明,最优方案中$c^0,\cdots,c^{k-1}$组合获得的总额不超过$c^k$. 我们假设$c^0,\cdots,c^{k-1}$都取上限$c-1$个,总和为$(c-1)(c^0+\cdots+c^{k-1})=(c-1)\frac{c^k-1}{c-1}=c^k-1<c^k$,因此我们要找零n元且方案最优,必须得取$c^k$面额的硬币.\par
下面我们用归纳法,归纳需要找零的面额n证明贪心策略是正确的. n=1时显然取能取的最大面额$c^0=1$是最优策略. 假设$n\leq t-1$时贪心均能取到最优策略,那么n=t时,由上述讨论,最优策略必须含有$2^k$. 由于$t-2^k\leq t-1$,由归纳假设,$t-2^k$用贪心法可以取到最优策略,而$t$的最优策略即为$t-2^k$的最优策略加上$2^k$(如果有比$t-2^k$的最优策略加上$2^k$更优的取t策略,那么t策略中除去$2^k$,得到的策略比$t-2^k$的贪心策略更优,与归纳假设矛盾);因此最优策略可以视为先贪心地取到$2^k$,再贪心地取剩下的$t-2^k$,也是贪心策略. 因此n=t时,贪心也能取到最优解. \par
综上,贪心是正确的.\par\ \ \par 

\textbf{c.} 设一组硬币面额为1,4,6,则8元按照贪心算法的取法为“6元$\times$1+1元$\times$2”,共3个硬币. 而实际最优策略为“4元$\times$2”,共2个硬币. 这说明了贪心并不能保证总能取到最优解.\par\ \ \par 

\textbf{d.} 这是一道完全背包问题,可以采用dp方法来做,递推式为a[i] = min(a[i], a[i-c[j]] + 1),其中a[i]表示i元最少可以用几个硬币凑出,c[j]表示给定的第j种硬币的面额. 为了使得每个金额可以多次取到,因此i从小到大进行循环. 伪代码如下,seq[i]是一个可重复集合,表示凑够i元的硬币取法:
\begin{lstlisting}
DP_ALGORITHM(n):
    let elements in a[i] be INF
    a[0] = 0
    seq[0] = {}
    for j from 0 to k - 1:
        for i from c[j] to n:
            if(a[i] > a[i - c[j]] + 1):
                a[i] = a[i - c[j]] + 1
                seq[i] = seq[i - c[j]]
                insert c[j] into seq[i]
    return seq[n]
\end{lstlisting}
很明显,两重循环,算法的时间复杂度为O(nk).

\end{document}
