\documentclass[UTF8]{ctexart}
\usepackage{amsmath}
\usepackage{diagbox}
\usepackage{textcomp}
\usepackage{graphicx}
\usepackage{float}
\usepackage{caption}
\usepackage{adjustbox}
\usepackage{subfigure}
\usepackage{geometry}
\usepackage{pifont}
\usepackage{gensymb}
\usepackage{bm}
\usepackage{tikz}
\usepackage{amstext}
\usepackage{amsfonts}

%引入代码块
\usepackage{listings}

\usepackage{xcolor}
%设置代码块格式

\definecolor{CPPGray}{RGB}{211,211,211}
\lstset{
 columns=fixed,       
 numbers=left,   % 在左侧显示行号
 numberstyle=\tiny\color{gray},% 设定行号格式
 frame=none,%none,% 不显示背景边框
 %aboveskip=1em,
 backgroundcolor=\color[RGB]{230,230,230},% 设定背景颜色
 keywordstyle=\color[RGB]{40,40,255},% 设定关键字颜色
 numberstyle=\footnotesize\color{darkgray},           
 commentstyle=\it\color[RGB]{0,96,96},% 设置代码注释的格式
 stringstyle=\rmfamily\slshape\color[RGB]{128,0,0},% 设置字符串格式
 showstringspaces=true,% 不显示字符串中的空格
 language=c++, % 设置语言
 morekeywords = {include,ull,int,double,return,static,typedef,if,else,for,long,void,class,struct,ll},                % 自加新的关键字(必须前后都是空格)
}

\begin{document}
\renewcommand{\thefootnote}{\fnsymbol{footnote}}
\newgeometry{left=2cm,bottom=3cm,right=2cm}
\linespread{1.4}
\title{\vspace{-5em}\heiti算法分析与设计基础\ \ 第十五周作业\vspace{-2.5em}}
\date{}
\maketitle
\begin{center}
{\fangsong 徐浩博\quad 软件02\quad2020010108}
\end{center}

\paragraph*{Problem\ \ 1\ \ }\ \par
算法如下:\par
辅助数/数组/链表:max\_num:当前剩余节点数最多的集合的剩余节点数;node\_list[v]:包含点v的所有set的链表;left\_node\_num[$S_i$]:$S_i$的剩余节点数;left\_num\_list[i]:剩余节点数为i的所有set的链表.

\begin{lstlisting}[mathescape=true]
for each $ S_i$ in F:
    for each v in $S_i$:
        node_list[v].insert($S_i$)
    left_node_num[$S_i$] = $|S_i|$
    left_num_list[$S_i$].insert($S_i$)
    max_num = max(max_num, $|S_i|$)

while max_num > 0:
    while true:
        if left_num_list[max_num] is empty:
            break
        head = left_num_list[max_num].head
        if left_node_num[head] != max_num:
            left_num_list[head].insert(head)
        else:
            for each v in head:
                for s in node_list[v]:
                    left_node_num[s] -= 1
                clear node_list[v]
        delete head from left_num_list[max_num]
    max_num -= 1
\end{lstlisting}
大概思路是:创建一个剩余节点数分别为1-max\_num这max\_num个链表,每个链表i内放置剩余节点数i的set;每次从max\_num中拿出一个头结点set,如果该set内剩余节点数已经变化,则将其插入对应的剩余节点数链表里,否则就要将该set内的点全部删去,同时更新各个其他set剩余节点数(注意,不是在剩余节点数链表中直接更新,而是用一个数组储存当前set剩余节点数).\par
该算法是$O(\sum_{S\in F}|S|)$的,原因是我们剩余节点数链表,每次对头结点进行操作:如果set内剩余节点数已经变化,那么操作复杂度是O(1)的,我们将这个复杂度归结到使set内剩余节点数变化的原因上(节点被删去);如果set内剩余节点仍是max\_num,那么就要删去该set内全部剩余节点,删除每个节点时需要更新所有包含该节点的set,每次更新O(1),这里可以将使set剩余节点数变化,头结点到达该set时需要O(1)更新的代价也加进来,一共也是O(1)的. 因此,总代价可以看成每个点在所有包含该节点的set内进行的O(1)操作,是$O(1)\times \sum_{S\in F}|S|=O(\sum_{S\in F}|S|)$的.

\paragraph*{Problem\ \ 2\ \ }\ \par
\begin{lstlisting}[mathescape=true]
APPROX-SUBSET-SUM(S, t, $\epsilon$):
    n = $|S|$
    $L_0$ = <0>
    for i = 1 to n:
        for each (e, set) in $L_{i-1}$:
            new_e = e + $x_i$
            pre[new_e] = $x_i$
            $L_i$.insert(new_e)
        $L_i$ = TRIM($L_i$, $\epsilon$/2n)
        remove from $L_i$ every element that is greater than t
    let $z^*$ be the largest value in $L_n$

    S = $\emptyset$
    while $z^*$ > 0:
        S.insert(pre$[z^*]$)
        $z^*$ = $z^*$ - pre[$z^*$]
    return S        
\end{lstlisting}
即我们记录$L_i$内每个新产生元素的前驱$x_i$,最后的集合如13-17行,通过前驱获知每一步加入的数. 显然,并没有改变原有程序的复杂度.

\end{document}

