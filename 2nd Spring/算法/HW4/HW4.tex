\documentclass[UTF8]{ctexart}
\usepackage{amsmath}
\usepackage{diagbox}
\usepackage{textcomp}
\usepackage{graphicx}
\usepackage{float}
\usepackage{caption}
\usepackage{adjustbox}
\usepackage{subfigure}
\usepackage{geometry}
\usepackage{pifont}
\usepackage{gensymb}
\usepackage{bm}
\usepackage{amstext}
\usepackage{amsfonts}
\usepackage{amssymb}
%引入代码块
\usepackage{listings}

\usepackage{xcolor}
%设置代码块格式

\definecolor{CPPGray}{RGB}{211,211,211}
\lstset{
 columns=fixed,       
 numbers=left,   % 在左侧显示行号
 numberstyle=\tiny\color{gray},% 设定行号格式
 frame=none,%none,% 不显示背景边框
 %aboveskip=1em,
 backgroundcolor=\color[RGB]{230,230,230},% 设定背景颜色
 keywordstyle=\color[RGB]{40,40,255},% 设定关键字颜色
 numberstyle=\footnotesize\color{darkgray},           
 commentstyle=\it\color[RGB]{0,96,96},% 设置代码注释的格式
 stringstyle=\rmfamily\slshape\color[RGB]{128,0,0},% 设置字符串格式
 showstringspaces=true,% 不显示字符串中的空格
 language=c++, % 设置语言
 morekeywords = {include,ull,int,double,return,static,typedef,if,else,for,long,void,class,struct,ll,while,do,throw,try,using,namespace,goto,switch,case,union,const,error,and,or,xor},                % 自加新的关键字(必须前后都是空格)
}

\begin{document}
\renewcommand{\thefootnote}{\fnsymbol{footnote}}
\newgeometry{left=2cm,bottom=3cm,right=2cm}
\linespread{1.4}
\title{\vspace{-5em}\heiti算法分析与设计基础\ \ 第四次作业\vspace{-2.5em}}
\date{}
\maketitle
\begin{center}
{\fangsong 徐浩博\quad 软件02\quad2020010108}
\end{center}

\subsection*{Problem 1}
\paragraph{a. }我们假设数组序号从1—n,n是数据的总数,那么我们对任意一个下标i的节点,有:\\ \\
父节点下标:\par
\begin{lstlisting}
PARENT(i):
    return  floor((i + d - 2) / d)
\end{lstlisting}
第k个子节点下标$(1\leq i \leq d)$:$di+k$\par
\begin{lstlisting}
kth-SON(i, k):
    return  (i - 1) * d + k + 1
\end{lstlisting}
\paragraph{b. }包含n个元素的d叉堆的高度是$\lfloor \log_d{n(d-1)}\rfloor$,规模是$\Theta(\log_d n)$.\\

\paragraph{c. }EXTRACT\-MAX

\begin{lstlisting}
EXTRACT-MAX:
    if heap-size[A] < 1:
        error "heap-underflow"
    max = A[1]
    A[1] = A[A.heap-size]
    A.heap-size = A.heap-size - 1
    MAX-HEAPIFY(A, 1)
    return max    
\end{lstlisting}
\begin{lstlisting}
MAX-HEAPIFY(A, i):
    MAX-SON = i
    for k = 1 to d:
        if(kth-SON(i, k) <= A.heap-size and A[kth-SON(i, k)] > A[MAX-SON]):
            MAX-SON = kth-SON(i, k)
    if MAX-SON != i:
        swap(A[i], A[MAX-SON])
        MAX-HEAPIFY(MAX-SON)
\end{lstlisting}
时间复杂度$O(d\log_d n)$.

\paragraph{d. }INSERT
\begin{lstlisting}
INSERT(key):
    A.heap-size = A.heap-size + 1
    A[A.heap-size] = key
    pos = A.heap-size
    while pos > 1:
        if A[pos] > A[PARENT(pos)]:
            swap(A[pos], A[PARENT(pos)])
            pos = PARENT(pos)
        else return
\end{lstlisting}
时间复杂度$O(\log_d n)$.

\paragraph{e. }INCREASE-KEY
\begin{lstlisting}
INCREASE-KEY(A, i, key):
    if A[i] > key:
        error "CANNOT-INCREASE"
    A[i] = key
    while i > 1:
        if A[i] > A[PARENT(i)]:
            swap(A[i], A[PARENT(i)])
            i = PARENT(i)
        else return
\end{lstlisting}
时间复杂度$O(\log_d n)$.

\subsection*{Problem 2}
\paragraph{a. }
考虑到数组中每个元素值都相同,那么随机取出任何一个数,数组中左右的数都会被分到PARTITION的左边,时间递推式可以写作:
$$T(n)=T(n-1)+\Theta(n)$$
那么则有$T(n)=\sum\limits_{i=1}^n=\frac{n(n+1)}{2}=\Theta(n^2)$,时间复杂度为$\Theta(n^2)$.

\paragraph{b. } PARTITION'(A, p, r)
\begin{lstlisting}[mathescape=true]
PARTITION$'$(A, p, r):
    x = A[r]
    q = p - 1
    t = p - 1
    for i = p to r - 1
        if A[i] < x:
            q = q + 1
            t = t + 1
            swap(A[i], A[q])
            swap(A[i], A[t])
        else if A[i] == x:
            t = t + 1
            swap(A[t], A[i])
    t = t + 1
    swap(A[r], A[t])
    return q, t
\end{lstlisting}
\paragraph{c. } RANDOMIZED-QUICKSORT'
\begin{lstlisting}[mathescape=true]
RANDOMIZED-QUICKSORT$'$(A, p, r):
    if p < r:
        q, t = PARTITION$'$(A, p, r)
        RANDOMIZED-PARTITION$'$(A, p, q)
        RANDOMIZED-PARTITION$'$(A, t + 1, r)
\end{lstlisting}
\paragraph{d. } 设原数组排好序后各个元素为$z_1\cdots z_n$,对任意两个元素$z_i\leq z_j$,显然如果要排序出$z_i,\ z_j$的大小,$Z_{ij}=\{z_q\cdots z_{i}\cdots z_j\cdots z_t\}$中必然有一元素与$z_i$或$z_j$作过比较($z_q$是等于$z_i$的下标最小的元素,$z_t$是等于$z_j$的下标最大的元素). 他们被选为主元pivot是等可能的. \par 1)选择$z_x(z_q\leq z_x\leq z_t,\ x\neq i,\ j)$作为pivot,这种情况发生的可能性是$\frac{t-q-1}{t-q+1}$,如果$z_x$与$z_i, z_j$均不相等时,则$z_i, z_j$彼此无须比较; 如果与$z_i, z_j$中的一个或两个相等时,根据RANDOMIZED-QUICKSORT'写法,$z_i, z_j$也无需比较.\par 2)选择$z_i$或$z_j$作为pivot,这种情况发生的可能性是$\frac{2}{t-q+1}$. 这种情况下$z_i, z_j$需要比较一次.\par
综合以上:$\displaystyle{Pr\{z_i\text{与}z_j\text{进行比较}\}=\frac{2}{t-q+1}\leq \frac{2}{j-i+1}}$,即比较可能性不大于7.4.2节结果,则总的期望比较次数也不大于7.4.2的结果,因此期望的运行时间不大于O(nlogn). 当重复元素较多,运行时间趋近于O(n);重复元素较少,运行时间则趋近于O(nlogn).

\end{document}
