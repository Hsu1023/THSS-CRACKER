\documentclass[UTF8]{ctexart}
\usepackage{amsmath}
\usepackage{diagbox}
\usepackage{textcomp}
\usepackage{graphicx}
\usepackage{float}
\usepackage{caption}
\usepackage{adjustbox}
\usepackage{subfigure}
\usepackage{geometry}
\usepackage{pifont}
\usepackage{gensymb}
\usepackage{bm}
\usepackage{amstext}
\usepackage{amsfonts}
%引入代码块
\usepackage{listings}

\usepackage{xcolor}
%设置代码块格式

\definecolor{CPPGray}{RGB}{211,211,211}
\lstset{
 columns=fixed,       
 numbers=left,   % 在左侧显示行号
 numberstyle=\tiny\color{gray},% 设定行号格式
 frame=none,%none,% 不显示背景边框
 %aboveskip=1em,
 backgroundcolor=\color[RGB]{230,230,230},% 设定背景颜色
 keywordstyle=\color[RGB]{40,40,255},% 设定关键字颜色
 numberstyle=\footnotesize\color{darkgray},           
 commentstyle=\it\color[RGB]{0,96,96},% 设置代码注释的格式
 stringstyle=\rmfamily\slshape\color[RGB]{128,0,0},% 设置字符串格式
 showstringspaces=true,% 不显示字符串中的空格
 language=c++, % 设置语言
 morekeywords = {include,ull,int,double,return,static,typedef,if,else,for,long,void,class,struct,ll},                % 自加新的关键字(必须前后都是空格)
}

\begin{document}
\renewcommand{\thefootnote}{\fnsymbol{footnote}}
\newgeometry{left=2cm,bottom=3cm,right=2cm}
\linespread{1.4}
\title{\vspace{-5em}\heiti算法分析与设计基础\ \ 第六周作业\vspace{-2.5em}}
\date{}
\maketitle
\begin{center}
{\fangsong 徐浩博\quad 软件02\quad2020010108}
\end{center}


\subsection*{Problem 1}
我们修改EXTENDED-BOTTOM-UP-CUT-ROD(p, n)
\begin{lstlisting}
EXTENDED-BOTTOM-UP-CUT-ROD(p, n, c):
	let r[0...n] and s[0...n] be new arrays
	r[0] = 0
	for j = 1 to n:
		q = p[j]
		s[j] = j
		for i = 1 to j - 1:
			if q < r[j - i] + p[i] - c:
				q = r[j - i] + p[i] - c
				s[j] = i
		r[j] = q
	return r and s
\end{lstlisting}
同时PRINT-CUT-ROD-SOLUTION保持不变:
\begin{lstlisting}
PRINT-CUT-ROD-SOLUTION(p, n, c):
	(r, s) = EXTENDED-BOTTOM-UP-CUT-ROD(p, n, c)
	while n > 0:
		print s[n]
		n = n - s[n]
\end{lstlisting}

\subsection*{Problem 2}
我们考虑四个矩阵相乘,$A_{6\times 5},\ B_{5\times 10},\ C_{10\times 8},\ D_{8\times 20}$,那么p数组就对应为:\\

\begin{table}[H]\begin{center}
\begin{tabular}{c|c|c|c|c}
	\hline
	p[0]&p[1]&p[2]&p[3]&p[4]\\
	\hline
	6&5&10&8&20\\
	\hline
\end{tabular}\end{center}
\end{table}
根据贪心的算法,p[0]p[1]p[4]最小,应该划分为(1,1),(2,4)两个子问题,(2,4)中又有p[1]p[3]p[4]最小,故划分为(1,3),(4,4)两个子问题,因此乘法顺序可以确定为$(A\times((B\times C)\times D))$,总计算规模为:
$$p[1]\times p[2]\times p[3]+p[1]\times p[3]\times p[4]+p[0]\times p[1]\times p[4]=1800$$
而根据动态规划的方法,最优的划分为$(A\times(B\times C))\times D$,总计算规模为:
$$p[1]\times p[2]\times p[3]+p[0]\times p[1]\times p[3]+p[0]\times p[3]\times p[4]=1600$$
由此可见,贪心算法并不一定总能给出最优解;上面举出的反例中,贪心求解只能求出次优解.
\end{document}
