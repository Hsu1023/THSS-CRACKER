\documentclass[UTF8]{ctexart}
\usepackage{amsmath}
\usepackage{diagbox}
\usepackage{textcomp}
\usepackage{graphicx}
\usepackage{float}
\usepackage{caption}
\usepackage{adjustbox}
\usepackage{subfigure}
\usepackage{geometry}
\usepackage{pifont}
\usepackage{gensymb}
\usepackage{bm}
\usepackage{tikz}
\usepackage{amstext}
\usepackage{amsfonts}

%引入代码块
\usepackage{listings}

\usepackage{xcolor}
%设置代码块格式

\definecolor{CPPGray}{RGB}{211,211,211}
\lstset{
 columns=fixed,       
 numbers=left,   % 在左侧显示行号
 numberstyle=\tiny\color{gray},% 设定行号格式
 frame=none,%none,% 不显示背景边框
 %aboveskip=1em,
 backgroundcolor=\color[RGB]{230,230,230},% 设定背景颜色
 keywordstyle=\color[RGB]{40,40,255},% 设定关键字颜色
 numberstyle=\footnotesize\color{darkgray},           
 commentstyle=\it\color[RGB]{0,96,96},% 设置代码注释的格式
 stringstyle=\rmfamily\slshape\color[RGB]{128,0,0},% 设置字符串格式
 showstringspaces=true,% 不显示字符串中的空格
 language=c++, % 设置语言
 morekeywords = {include,ull,int,double,return,static,typedef,if,else,for,long,void,class,struct,ll},                % 自加新的关键字(必须前后都是空格)
}

\begin{document}
\renewcommand{\thefootnote}{\fnsymbol{footnote}}
\newgeometry{left=2cm,bottom=3cm,right=2cm}
\linespread{1.4}
\title{\vspace{-5em}\heiti 形式语言与自动机\vspace{-2.5em}}
\date{}
\maketitle
\begin{center}
{\fangsong 徐浩博\quad 软件02\quad2020010108}
\end{center}\begin{center}
\begin{tabular}{c|l|l}
    \hline
    DFA&五元组$\{Q,\Sigma,\delta,q_0,F\}$&$q_0\in Q, F\subseteq Q,\delta:Q\times \Sigma\to Q$\\
    \hline
    NFA&五元组$\{Q,\Sigma\cup\{\epsilon\},\delta,q_0,F\}$&$q_0\in Q, F\subseteq Q,\delta:Q\times \Sigma\cup\{\epsilon\}\to 2^Q$\\
    \hline
    文法&四元组$\{V,T,S,P\}$&Variant、Terminant、Start、Production,$V\cap T=\emptyset$\\
    \hline
    PDA&七元组$\{Q,\Sigma,\Gamma,\delta,q_0,Z_0,F\}$&$\Sigma$输入符号,$\Gamma$栈符号,$\delta:Q\times (\Sigma\cup \{\epsilon\})\times \Gamma \to 2^{Q\times \Gamma^*}$\\
    \hline
    空栈型PDA&六元组$\{Q,\Sigma,\Gamma,\delta,q_0,Z_0\}$&$\Sigma$输入符号,$\Gamma$栈符号,$\delta:Q\times (\Sigma\cup \{\epsilon\})\times \Gamma \to 2^{Q\times \Gamma^*}$\\
    \hline
    图灵机&七元组$\{Q,\Sigma,\Gamma,\delta,q_0,B,F\}$&$\Sigma\subset \Gamma, B\in \Gamma-\Sigma,\delta:Q\times \Gamma \to Q\times \Gamma \times \{L,R\}$
\end{tabular}\end{center}
\newpage
NFA转DFA:\\
\begin{figure*}[htbp]
\includegraphics*[scale = 1]{NFAtoDFA.PNG}
\includegraphics*[scale = 1]{NFAtoDFA2.PNG}
\end{figure*}

\end{document}

