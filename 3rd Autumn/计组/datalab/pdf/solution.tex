\documentclass[UTF8]{ctexart}
\usepackage{amsmath}
\usepackage{diagbox}
\usepackage{textcomp}
\usepackage{graphicx}
\usepackage{float}
\usepackage{caption}
\usepackage{adjustbox}
\usepackage{subfigure}
\usepackage{geometry}
\usepackage{pifont}
\usepackage{gensymb}
\usepackage{bm}
\usepackage{tikz}
\usepackage{amstext}
\usepackage{amsfonts}
%引入代码块
\usepackage{listings}

\usepackage{xcolor}
%设置代码块格式

\definecolor{CPPGray}{RGB}{211,211,211}
% \lstset{
%  columns=fixed,       
%  numbers=left,   % 在左侧显示行号
%  numberstyle=\tiny\color{gray},% 设定行号格式
%  frame=none,%none,% 不显示背景边框
%  %aboveskip=1em,
%  backgroundcolor=\color[RGB]{230,230,230},% 设定背景颜色
%  keywordstyle=\color[RGB]{40,40,255},% 设定关键字颜色
%  numberstyle=\footnotesize\color{darkgray},           
%  commentstyle=\it\color[RGB]{0,96,96},% 设置代码注释的格式
%  stringstyle=\rmfamily\slshape\color[RGB]{128,0,0},% 设置字符串格式
%  showstringspaces=true,% 不显示字符串中的空格
%  language=c++, % 设置语言
%  morekeywords = {include,ull,int,double,return,static,typedef,if,else,for,long,void,class,struct,ll, unsigned},                % 自加新的关键字(必须前后都是空格)
% }

\begin{document}

\renewcommand{\thefootnote}{\fnsymbol{footnote}}
\lstset {
    language=C,
    frame=tb,
    tabsize=4,
    showstringspaces=false,
    numbers=left, 
    breaklines=true,
    %upquote=true,
    commentstyle={\color[RGB]{0,96,96}},
    % commentstyle={\color{black}},
    keywordstyle=\color{blue},
    stringstyle=\color{red},
    basicstyle=\small\ttfamily\songti, % basic font setting
    emph={int,char,double,float,unsigned,void,bool},
    emphstyle={\color{blue}},
    % escapechar=\&,
    % keyword highlighting
    classoffset=1, % starting new class
    otherkeywords={>,<,.,;,-,!,=,~,|,&,^,+,-},
    morekeywords={>,<,.,-,!,=,~,|,&,!,^,+,-},
    keywordstyle=\color{red},
    classoffset=0,
    % frame=single, % 边框
}
\lstset{escapeinside=``}
\newgeometry{left=2cm,bottom=3cm,right=2cm}
\linespread{1.4}
\title{\vspace{-5em}\heiti计算机组成原理\ \ DataLab Report\vspace{-2.5em}}
\date{}
\maketitle
\begin{center}
{\fangsong 徐浩博\quad 软件02\quad2020010108}
\end{center}

\subsection*{1.\ Bit Manipulations}
\paragraph*{1.1\ bitAnd(x,\ y)}\ \par
由逻辑运算的摩根定律,$x \wedge y = \neg (\neg x \vee \neg y)$,转换为数字运算即$x\& y=~((~x)|(~y))$

\begin{lstlisting}
    int bitAnd(int x, int y) {
        return ~((~x) | (~y));
      }
\end{lstlisting}

\paragraph*{1.2\ getByte(x,\ n)}\ \par
首先n是位数,我们通过乘3(右移3位)获得比特数(int shift),然后将0xff左移shift位与x做AND操作获得mask的所求比特,然后再向右移位获得最终结果。注意最后移位完成后仍需AND 0xff以避免负数算术右移使得高位补1的情况。
\begin{lstlisting}
    int getByte(int x, int n) {
        int shift = n << 3;
        int result_with_shift = x & (0xff << shift);
        // get Byte with shift
        return (result_with_shift >> shift) & 0xff;
        // AND 0xff to prevent arithmetic right shift of negative number
    }
\end{lstlisting}

\paragraph*{1.3\ logicalShift(x,\ n)}\ \par
首先对于正数,逻辑右移和算术右移等价;对于负数,逻辑右移n位实际上将$$x=-TMAX+\alpha_1 TMAX/2+\alpha_2 TMAX/4+\cdots$$
移动成了$$x_{logical}=(TMAX/{2^n})+\alpha_1 TMAX/{2^{n+1}}+\alpha_2 TMAX/{2^{n+2}}+\cdots$$
其中$\alpha_i=0\ or\ 1$,但算术右移实际上是
$$x_{arithmetic}=(-TMAX+TMAX/2+\cdots+TMAX/2^{n})+\alpha_1 TMAX/{2^{n+1}}+\alpha_2 TMAX/{2^{n+2}}+\cdots$$
显然给算术右移结果加$TMAX/2^{n+1}$即为逻辑右移结果,而$TMAX/2^{n+1}=2^{32 - n}=2^{32 +(\sim n) +1}=2^{33+(\sim n)}$,因此我们只需要在判别$n!=0 \&\& x < 0$的基础上加$q=1<<(33+(\sim n))$即可.
\begin{lstlisting}
    int logicalShift(int x, int n) {
        int n_not_zero = (n & 1) | ((n >> 1) & 1) | ((n >> 2) & 1) | ((n >> 3) & 1) | ((n >> 4) & 1);
        // = 1, if n != 0; = 0, else
        int n_not_zero_and_x_negative = n_not_zero & (x >> 31);
        // = 1, if n != 0 && x < 0; = 0, else
        int q = n_not_zero_and_x_negative << (33 + (~n));
        // if n != 0 && x < 0, then add 2^(32 - n); else add 0
        return (x >> n) + q;
      }
\end{lstlisting}

\paragraph*{1.4\ bitCount(x)}\ \par
本题的基本思路是先两两计算bitcount,以二进制的方法记录到原位上,然后合并到四位,再合并到八位,以此类推.\par
为了避免负数的算术右移,我们先提取符号位,并且给x AND 0x7fffffff以保证bitCount计数时是对正数运算,最后答案在加上符号位是否为1即可.\par
其次有如下观察:对于2位的二进制数,计算bitcount可以用以下方式:$$Bitcount(a)=BitCount((a_1a_2)_2)=(a_1a_2)_2-(a_1)_2=a - (a>>1)$$
推广到32位int,两两计算BitCount,只需要借助一个$mask=(0101...01)_2=0x55555555$,由$x-(x\&mask)$得到. 这样每两位二进制数的1的个数就以二进制数的方式记录到了原位.\par
再次,我们要将2组2位数的计算拼成1组4位数,这是简单的,我们只需要借助一个$mask=(0011...0011)_2=0x33333333$,由$(x\& mask)+((x>>2)\&mask)$得到. \par
然后,我们要将2组4位数的计算拼成1组8位数,借助$mask=(00001111...00001111)_2=0x0f0f0f0f$,由$(x\& mask)+((x>>4)\&mask)$得到,为了节省运算符,我们可以将其写做$(x+(x>>4))\&mask$,能够这么运算的原因是8位数最多8个1,1的个数可以用4个二进制数表示,故最终结果只有后四位有效,并不会溢出到高四位上,也因此,可以在加法之后统一AND mask.\par
之后,将2组8进制数的计算结果拼成1组16进制数,这一步可以不需要mask的帮助,原因是如上所述,一组8位数的高4位为0。因此直接可以由$(x+(x>>8))$得到。\par
最后2组16进制数拼成1个32进制数,加上第二段所叙述的符号位即为最终结果。
\begin{lstlisting}
    int bitCount(int x) {
        int sign, n_0x7fffffff, n_0x55, n_0x33, n_0x0f;
      
        // process signal of x
        sign = (1 & (x >> 31));
        n_0x7fffffff = (1 << 31) + (~ 1) +1;
        x = x & n_0x7fffffff;
      
        // 32 group * 1 bit/group -> 16 group * 2 bit/group 
        n_0x55 = 85 + (85 << 16); // 85 = 0x55
        n_0x55 = n_0x55 + (n_0x55 << 8); // mask = 0x55555555
        x = x + (~((x >> 1) & n_0x55)) + 1; // x = x - (x & mask)
      
        // 16 group * 2 bit/group -> 8 group * 4 bit/group 
        n_0x33 = 51 + (51 << 16); // 51 = 0x33
        n_0x33 = n_0x33 + (n_0x33 << 8); // mask = 0x33333333
        x = (x & n_0x33) + ((x >> 2) & n_0x33); // x = (x & mask)+((x >> 4) & mask)
      
        // 8 group * 4 bit/group -> 4 group * 8 bit/group 
        n_0x0f = 15 + (15 << 16); // 51 = 0x0f
        n_0x0f = n_0x0f + (n_0x0f << 8); // mask = 0x0f0f0f0f
        x = (x + (x >> 4)) & n_0x0f; // x = (x + (x >> 8)) & mask
      
        // 4 group * 8 bit/group -> 2 group * 16 bit/group 
        x = x + ((x >> 8));// 16
      
        // 2 group * 16 bit/group -> result
        return (x & 0xff) + ((x >> 16) & 0xff) + sign;
      }
\end{lstlisting}

\paragraph*{1.5\ bang(x)}\ \par
此题与上题类似,可以采用二分法. 将32bit通过高16bit OR 低16bit操作保留16bit,但并不丢失原32bit是否有1这一信息(若高16bit与低16bit对应位有一个1,OR后的数也保留有1). 以此类推,16bit通过高8bit OR 低8bit保留8bit,直到最终得到1bit,
该bit=1则表示原数有1,否则无1. 在这一过程中,需要注意OR采用x | (x \textgreater\textgreater n)的方式,
但这一方式仍然在高位保留了原来的数,因此最终需要给x \& 1,得到的就是最终1bit,
高位确保全为0. 我们最终需要判断!x,当且仅当无1时!x=1,否则!x=0,因此,只需要(x \& 1) \^{} 1就能得到最终结果.

\begin{lstlisting}
    int bang(int x) {
        x = x | (x >> 16); // 32 -> 16
        x = x | (x >> 8); // 16 -> 8 
        x = x | (x >> 4); // 8 -> 4 
        x = x | (x >> 2); // 4 -> 2 
        x = x | (x >> 1); // 2 -> 1 
        return (x & 1) ^ 1;
      }
\end{lstlisting}

\subsection*{2.\ Two’s Complement Arithmetic}
\paragraph*{2.1\ tmin()}\ \par
Tmin即最高位为1,其余为全为0,因此可以通过(1\textless\textless 31)得到.
\begin{lstlisting}
    int tmin(void) {
        return (1 << 31);
      }
\end{lstlisting}

\paragraph*{2.2\ fitsBits(x,n)}\ \par
对于n bits,我们能够表示数的范围是$-2^{n-1}\sim 2^{n-1}-1$,由于正负数不一致,因此我们需要分情况讨论.\par
对于非负数,我们只需要将其右移(n-1)位,若其不为0,则不能被表示,反之为能表示.\par
对于负数,我们知道$|x|=-x=(\sim x) + 1$,因此$\sim x = |x| - 1$. 我们再由n bits能表示的范围$-2^{n-1}\sim 2^{n-1}-1$可知,对于负数x,x能用n位表示当且仅当|x|-1能够被表示,也即$\sim x$能被表示. 因此我们将x按位取反然后按照非负数的操作,右移(n-1)为看其是否为0.\par
结合以上两点,我们可以先计算出x的符号sign,然后将分情况讨论写成逻辑运算符,负数的时候sign为真,AND上对负数的判断;非负数!sign为真,AND上对非负数的判断;最后将两种结果OR起来即可.
\begin{lstlisting}
  int fitsBits(int x, int n) {
    int sign, positive_fitBits, negative_fitBits;
    sign = (1 & (x >> 31));
    // sign = 1, if x is negative; sign = 0, else

    n = n + (~1) + 1;
    // n = n - 1

    positive_fitBits = !((x >> n)) & (!sign);
    // positive_fitBits = 1 iff. x >= 0 and x satisfies fitsBits 
    negative_fitBits = (!((~x) >> n)) & sign;
    // negative_fitBits = 1 iff. x >= 0 and x satisfies fitsBits 
    return positive_fitBits | negative_fitBits;
    }
\end{lstlisting}

\paragraph*{2.3\ divpwr2(x,\ n)}\ \par
对于正数,我们直接右移n位即可;而对于负数,算术右移导致向下舍入,并不是我们期望的向上舍入,因此我们需要加一个偏置$2^n-1$后右移. 下面我们说明这种偏置是正确的.\par
1)假设x能够被2\^{}n整除,那么$x/2^n=\lfloor x/2^n \rfloor=\lfloor (x+2^n-1)/2^n$,因此偏置并不会改变最终右移n位的结果. 在这里注意n=0时实际上偏置也等于0,因此也成立.\par 2)假设x不能被2\^{}n整除,我们所求的结果$\lceil x/2^n \rceil= \lfloor x/2^n \rfloor +1$,而不能整除意味着它的余数必然大于1,加上$2^n-1$会导致除以$2^n$的时候整数部分变大1,因此$\lfloor (x+2^n-1)/2^n \rfloor = \lfloor x/2^n \rfloor +1$,从而我们所求的结果也可以用$\lfloor (x+2^n-1)/2^n \rfloor$表示. 综合以上两点,我们就证明了这种做法的正确性.\par
回到运算当中,我们先用sign记录x的正负(负为1),然后将sign右移n位减sign计算出add偏置量:当x非负时sign等于0,实际上偏置量等于0,而x为负数时偏置量为$2^n-1$. 最后加上偏置量右移n位即可.

\begin{lstlisting}
    int divpwr2(int x, int n) {
        int sign = (x >> 31) & 1;
        // sign = 1, if x is negative; sign = 0, else
        int add = (sign << n) + (~sign) + 1; 
        // plus 2 ^ n - 1
        return (x + add) >> n; 
        // right shift n bits
      }
\end{lstlisting}

\paragraph*{2.4\ negate(x)}\ \par
实际上这道题考察原码和补码的相互转换,假设x是原码,x'是补码,那么有x+x'=0xffffffff,则有x+x'+1=0,因此$-x=x'+1=\sim x + 1$,且有$-x'=x+1=\sim x' + 1$. 因此无论是对负数还是对正数,取相对数都只需要求$\sim x+1$.
\begin{lstlisting}
    int negate(int x) {
        return ~x + 1;
      }
\end{lstlisting}

\paragraph*{2.5\ isPositive(x)}\ \par
我们首先看x的符号位sign = 1 \& (x \textgreater\textgreater 31),!sign=1当且仅当x非负,但这里还需要考虑x=0的情况. !!x=1当且仅当x非0,因此结合二者,(!sign) \& (!!x)当且仅当x大于0.
\begin{lstlisting}
    int isPositive(int x) {
        int sign = 1 & (x >> 31);
        // sign = 1, if x is negative; sign = 0, else
      
        return (!sign) & (!!x);
        // !sign = 1, if x >= 0; sign = 0, else
        // !!x = 1, if x != 0; !!x = 0, else
        // (!sign) & (!!x) iff. x > 0
      }
\end{lstlisting}

\paragraph*{2.6\ isLessOrEqual(x,\ y)}\ \par
x \textless= y 有两种情况,我们根据他们的正负号分别讨论:\par
1)x \textless 0 ,在这种情况下要满足less or equal只能有x为负数但y非负,我们可以通过查看x,y的二进制数符号位很方便地获知;\par
2)x \textgreater= 0 ,在这种情况下y只能非负,而且y \textgreater= x. y非负很好判断,我们只需要判断x - y \textless= 0,方法是查看x - y的符号位(由于没有减号,故只能手动补码运算),注意在x,y都为非负数的情况下,减法不会超出int的范围.\par
以上两点包含了x \textless= y的所有可能,只需满足一点即可得知isLessOrEqual(x, y) = 1,反之,不在以上情况内必有isLessOrEqual(x, y) = 0.
\begin{lstlisting}
    int isLessOrEqual(int x, int y) {
        // case 1:
        int sg1 = (1 & (x >> 31));
        // sg1 = 1, if x is negative; sign = 0, else
        int sg2 = (1 & (y >> 31));
        // sg2 = 1, if y is negative; sign = 0, else
        int case1 = sg1 & (!sg2);
        // x < 0 && y >= 0
      
        // case 2:
        int sub = x + (~y) + 1;
        // sub = x - y
        int sg_sub = (1 & (sub >> 31));
        // sg_sub = 1, if sub is negative; sign = 0, else
        int case2 = (!(sg1 ^ sg2)) & (sg_sub | (!sub));
        // x, y have same sign, and (x - y < 0 or x - y == 0)
        return case1 | case2;
      }
\end{lstlisting}

\paragraph*{2.7\ ilog2(x)}\ \par
此题我们可以采用二进制表示数的方法,我们已知ilog(x)不超过32,可以用5位二进制数表示,而\textbf{二进制数的表示方法唯一},这是我们解题的关键点.\par
我们先查看ilog的第5位二进制数,方法是将x右移$2^5=16$位,若x仍不为0,说明ilog大于$2^5=16$,因此ilog的第5位二进制数必为1(二进制数的表示方法唯一). 注意,如果第5位数为1,则需令$x >>= 16$,如果第5位数不为1,则需令$x >>= 0$,即不变,综合起来可以表示为$x >>= (16 \& ((\sim exist) + 1))$,其中后面一项exist代表ilog的第5位二进制数,$(\sim exist) + 1=0xffffffff, if\ exist\ =\ 1;\ (\sim exist) + 1=0, else$.\par
与之类似,对于ilog的第4位二进制数,将x右移8位查看,方法同上;以此类推,直到ilog的第1位二进制数.\par
最后,我们根据每一位二进制数,将各位乘以权重,最后就可以得到ilog的值了.
\begin{lstlisting}
    int ilog2(int x) {
        int x_16, x_16_exist;
        int x_8, x_8_exist;
        int x_4, x_4_exist;
        int x_2, x_2_exist;
        int x_1, x_1_exist;
      
        x_16 = x >> 16;
        x_16_exist = !!(x_16);
        // test whether x >> 16 == 0, which is equal to the fifth bit number of ilog2 
        x = x >> (16 & ((~x_16_exist) + 1));
        // x = x >> 16 if x >> 16 != 0, else x = x
        
        x_8 = x >> 8;
        x_8_exist = !!(x_8);
        // test whether x >> 8 == 0, which is equal to the fourth bit number of ilog2 
        x = x >> (8 & ((~x_8_exist) + 1));
        // x = x >> 8 if x >> 8 != 0, else x = x
      
        x_4 = x >> 4;
        x_4_exist = !!(x_4);
        // test whether x >> 4 == 0, which is equal to the third bit number of ilog2 
        x = x >> (4 & ((~x_4_exist) + 1));
        // x = x >> 4 if x >> 4 != 0, else x = x
      
        x_2 = x >> 2;
        x_2_exist = !!(x_2);
        // test whether x >> 2 == 0, which is equal to the second bit number of ilog2 
        x = x >> (2 & ((~x_2_exist) + 1));
        // x = x >> 2 if x >> 2 != 0, else x = x
      
        x_1 = x >> 1;
        x_1_exist = !!(x_1);
        // test whether x >> 2 == 0, which is equal to the second bit number of ilog2 
      
        return x_1_exist + (x_2_exist << 1) + (x_4_exist << 2) +
         (x_8_exist << 3) + (x_16_exist << 4);
        // get value of ilog2 by bit numbers
      }
\end{lstlisting}

\subsection*{3.\ Floating-Point Operations}
\paragraph*{3.1\ float\_neg(uf)}\ \par
此题主要分三种情况讨论:\par
1)NaN不合法:当exponent=0xff(加上偏移量为0x7f800000)且fraction!=0,直接返回原数uf;\par
2)uf非负:uf最高位为0,则直接改变最高位为1,方法是给原数OR 0x80000000;\par
3)uf负:uf最高位为1,则直接改变最高位为0,方法是给原数AND 0x7fffffff;\par
\begin{lstlisting}
    unsigned float_neg(unsigned uf) {
        unsigned fraction = uf & 0x007fffff;
        unsigned exponent = uf & 0x7f800000;
        unsigned sign;
      
        // NaN
        if (fraction != 0 && exponent == 0x7f800000)
          return uf; 
        sign = (uf >> 31) & 1;
      
        // not negative
        if(sign){
          return uf & 0x7fffffff;
        }
        // negative
        else{
          return uf | 0x80000000;
        }
      }
\end{lstlisting}

\paragraph*{3.2\ float\_i2f(x)}\ \par
首先特判x=0,直接返回0即可. \par
然后我们可以观察,x除去符号位的二进制表示事实上通过简单的移位和舍入即可作为fraction part,其中移位的过程记录下来就可以变成exponent part. 具体来说,我们提取x符号位,并且将x统一转化为正数形式,这里注意特判x=0x800000000,-x也等于0x800000000,事实上通过移位,最终结果也是正确的,因此我们对于负数x统一取x=-x.\par 下面我们需要统计二进制数种最高位1的位置,这里通过一个while循环来实现. 对于最高位1低于或等于23位的,即不足fraction部分23位的,我们需要给末位补零;而高于23位的,我们需要进行舍入. 我们首先获得舍入的边界情况(即code中的limit\_case),该情况中整数部分与x右移结果一致,而小数部分为1000...,即为十进制中的.5. x与边界情况比较. 向上舍入只有两种情况:1)若大于边界情况,则小数部分大于>.5,向上舍入;2)若等于边界情况,且整数部分为奇数(右移后二进制末位为1),则满足向偶数的向上舍入. 除此之外,均为向下舍入,直接保留整数部分. 特别注意,向上舍入的时候fraction part可能会恰好溢出23位,此时应该令fraction part=0(即直接保留后23位)并给E加一.\par
最后,exponent=bias+E=127+E

\begin{lstlisting}
  unsigned float_i2f(int x) {
    unsigned sign, temp, limit_case, E, fraction, exponent;
    // if x == 0
    if(!x)
        return 0;
    
    sign = (x >> 31) & 1;
    // sign = 1, if x is negative; sign = 0, else
    if(sign)
        x = -x; 
    // if x is negative, turn it to positive
    temp = x;
    E = 0xffffffff; //E = -1

    // test the position of highest 1 
    while (temp){
        temp = temp >> 1;
        E = E + 1;
    }

    // if the position is lower than 24, left shift
    if(E <= 23){
        temp = x << (23 - E);
    }
    // if the position is higher than 24, right shift
    else{
        temp = x >> (E - 23);
        limit_case = ((temp << 1) | 1) << (E - 24); // limit case of rounding (0.5)
        temp = temp & 0x7fffff; // reserve the fraction part
        if (x > limit_case) // if greater than limit case (> 0.5)
        temp = temp + 1;
        else if(x == limit_case && (temp & 1)) // if equal to limit case AND satisfying Ties To Even (= 0.5) 
        temp = temp + 1;
        if (temp & 0x800000) // if fraction part overflow, then div by 2 and exponent ++
        E++;
    }

    fraction = temp & 0x7fffff; // get fraction part
    exponent = 127 + E; // get exponent part
    return (sign << 31) + (exponent << 23) + fraction;
    }
\end{lstlisting}

\paragraph*{3.3\ float\_twice(uf)}\ \par
我们可以将uf分为以下情况讨论:\par
1)NaN以及inf,此种情况exponent=1111 1111,带shift为0x7f800000,乘2将返回原数;\par
2)乘2溢出,此种情况exponent=1111 1110,带shift为0x7f000000,溢出时符号位不变,exponent位全1(OR 0x7f8000000),fraction位全0(AND 0xff800000);\par
3)非归约形式,此种情况exponent=0000 0000,考虑到非归约形式和归约形式之间的gap是连续且等距的,因此保留符号位,直接将exponent和fraction部分合并起来左移一位即可;\par
4)归约形式,此种情况直接将exponent加一即可,考虑到带shift,因此将exponent左移23位加一再右移23位,和fraction、sign拼合即可.
\begin{lstlisting}
  unsigned float_twice(unsigned uf) {
    unsigned fraction = uf & 0x007fffff; // get fraction part
    unsigned exponent = uf & 0x7f800000; // get fraction part with shift
    unsigned sign = uf & 0x80000000; // get sign part with shift

    if (exponent == 0x7f800000)
        return uf; // NaN & inf

    if (exponent == 0x7f000000)
        return (uf | 0x7f800000) & 0xff800000; // mul 2 then overflow, return inf

    if (exponent == 0)
        return sign | (uf << 1); // De-normalized value

    return (sign + (((exponent >> 23) + 1) << 23) + fraction); // Normalized Value
}

\end{lstlisting}

\subsection*{4.\ Summary}
我完成了本次实验的全部内容,总共用了220个operators,并在自动脚本测试中获得了全部的Corr和Perf分数.\par
在本次实验中,我对于整型、浮点数有了更深入的了解,尤其是浮点数部分,通过一系列的lab,我对于其表示、几种特殊情况、舍入等等有了更全面更细致的掌握. 尤其是限定运算符号集合以及使用个数以后,我更娴熟地掌握了各种运算符的运用,也对于算术右移等操作符有了进一步的理解.\par
除此之外,我还充分利用了Mask、二分、二进制表示等运算/算法技巧. 总体来说,题面易于理解,形式也很新颖,但有些题目仍有不小难度,解决不定令人抓耳挠腮,忽然突破又让人会心一笑,别有趣味.

\end{document}

